\chapter{Zusammenfassung und Aussichten}
\label{cha:zusammenfassung}

Da alle Themenblöcke der Arbeit somit abgearbeitet wurden, kann im Folgenden ein finaler Ausblick in die Zukunft dieses Themengebiets gegeben werden. Auch wird die gesamte Arbeit abschließend zusammengefasst.

\section{Aussicht für die Zukunft}

Das in dieser Arbeit vorgestellte und behandelte Themengebiet steckt trotz jahrelanger Forschung immer noch in den Kinderschuhen. Die DNA-Tile-basierte Self-Assembly wird fast ausschließlich auf dem Papier und in der Theorie diskutiert und analysiert, da praktische Tests und Versuche aufwendig und teuer sind.

Trotzdem lässt sich in den präsentierten theoretischen Modellen das Potenzial dieser Technologie erkennen. DNA ermöglicht einen Bottom-Up Ansatz für Nanostrukturen. Durch Arbeiten wie diese wird klar, dass mit DNA-Tiles nicht nur komplexe Berechnungen möglich sind, sondern auch komplexere Kommunikation. Die Vorstellung, auf diese Weise Nanonetzwerke zu implementieren, wird immer greifbarer. Besonders für den Bereich der Medizin ist dies von großem Interesse. Ein In-Body-Kommunikationssystem, das Krankheiten frühzeitig identifizieren und lokal behandeln kann, rückt durch kontinuierliche und intensive Forschung im Bereich der DNA-Tiles immer näher.

Die Medizin ist nur eines von vielen potenziellen Anwendungsfeldern. Auf Nanoebene gibt es einige vielversprechende Möglichkeiten. Mit weiterer Forschung könnten Nanonetzwerke in verschiedenen Bereichen wertvolle Beiträge leisten.

Wenn auf dieser Arbeit aufbauend weiter geforscht werden soll, dann wäre zum Beispiel eine Möglichkeit, die Simulationsergebnisse durch Simulationen in kTHAM zu erweitern. Dies war aus Gründen der Rechenleistung in dieser Arbeit nicht möglich. Ein Vergleich der Ergebnisse zwischen kTAM und kTHAM wäre jedoch interessant und könnte bessere Einblicke in die Umsetzbarkeit der Mechanismen bieten. 

Auch wurden einige Konzepte offengelassen. Der Dialogaufbau, das Routing und Framing wurden beispielsweise in dieser Arbeit als unnötig oder inhärent in DNA-Tile-basierter Self-Assembly beschrieben. Ist ein Mechanismus dieser Art jedoch für ausgewählte Anwendung notwendig, könnten weitere Implementierungen vorgestellt werden, die noch spezifischere Aspekte von Kommunikationsmechanismen in Netzwerken umsetzen. 

In dieser Arbeit wurde oft über den medizinischen Kontext gesprochen. Es wurde oftmals angenommen, dass so ein System im Blutkreislauf eines Menschen implementiert werden soll. Doch die Simulationen und die Modelle basieren alle auf rein mathematischer Ebene und betrachten nicht den Unterschied zwischen in-vivo und in-vitro. Ein darauf fokussiertes Modell könnte weitere Informationen für die in dieser Arbeit entwickelten Mechanismen bringen. 

Auch wurde in den Grundlagen angedeutet, dass einige Mechanismen sinnvoller auf Mikroebene durchgeführt werden könnten. Als Beispiel wurde ein Body-Area-Netzwerk genannt, das die zentrale Steuerung des Systems übernimmt. Für diese Arbeit wurde immer angenommen, dass Mechanismen existieren, die DNA-Tiles bilden und kontrolliert freilassen. Weitere Forschung könnte sich mit solchen Systemen befassen. 

\section{Zusammenfassung}
Diese Arbeit hat sich tiefgründig mit DNA-Tile-basierten Nanonetzwerken durch Tiles und deren Self-Assembly befasst. Nach der Einleitung wurden dafür die Grundlagen angefangen bei der Desoxyribonukleinsäure, über die Tilebildung, bis hin zu Self-Assembly und den Tile-Assembly Modellen dargelegt. Auch setzen sich die Grundlagen in der Arbeit mit Mechanismen auseinander, die in herkömmlichen Netzwerken durch Protokolle definiert werden. Zusätzlich zu den Grundlagen stellt diese Arbeit drei verwandte Publikationen vor, die genauer vorgestellt wurden. Jedoch gibt es keine Arbeiten, die sich mit dem konkreten Problem befassen, das in dieser Arbeit behandelt wird. Übersetzung von Kommunikationsprotokollen und ihren Mechanismen auf Nanoebene wird in den meisten wissenschaftlichen Publikationen in anderen Nanosystemen durchgeführt.

In dieser Arbeit wurde ein umfassendes Konzept für eine Vielzahl von Mechanismen vorgestellt, darunter Adressierung, Fehlererkennung, Fehlerkorrektur, Framing, Datenflusskontrolle, Nachrichtencodierung und Flags. Zu jedem dieser Themen wurden sowohl die Herausforderungen als auch die Implementierungsansätze beschrieben. Nach Möglichkeit wurden diese Mechanismen in einem Python-Skript umgesetzt\marginnote{\qrcode[height=1cm]{https://github.com/Falkenheim/Tile-Generator}}, um die in der Simulationsumgebung NetTAS erstellten Tilesets zu erweitern.

Im Rahmen dieser Arbeit wurden die vorgestellten Mechanismen mithilfe der Simulationsumgebung NetTAS untersucht und evaluiert. Die Ergebnisse zeigen, dass bei einigen Mechanismen der Self-Assembly-Prozess durch den jeweiligen Ansatz deutlich verlängert wird. Dennoch können einige dieser Mechanismen in bestimmten Anwendungen essenziell sein. Ein Limitierungsfaktor dieser Untersuchung war, dass aufgrund der zur Verfügung stehenden Rechenkapazitäten das komplexere und realitätsnähere Simulationsmodell kTHAM nicht eingesetzt werden konnte. Stattdessen basieren alle Simulationsergebnisse ausschließlich auf dem Tile-Assembly-Modell kTAM. Empfehlungen für zukünftige Forschungsansätze im Kontext dieser Limitation und darüber hinaus finden sich im Ausblick dieses Kapitels.



%!TEX root = thesis.tex
%\chapter{Zusammenfassung und Ausblick}
%\label{cha:zusammenfassung}

%In \Cref{cha:einleitung} wurde...

%In \Cref{sec:holstentor} ...

%In \Cref{sec:cite} ...
