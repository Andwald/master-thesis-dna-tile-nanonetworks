\chapter{Besprechungen}

\section{Besprechung 27-02-23}
Fragen:
\begin{itemize}
  \item Sprache der Arbeit
  \begin{description}
    \item[Antwort: ] deutsch
  \end{description}
  \item Anmeldedatum noch etwas aufschieben (eine Woche maximal 2) smart?
  \begin{description}
    \item[Antwort: ]
  \end{description}
  \item Reihenfolge des Vorgehens bei Kapiteln checken
  \begin{enumerate}
    \item Grundlagen, die 100$\%$-ig benutzt werden
    \item Related Works
    \item Hauptkapitel / Simulationen
    \item Grundlagen, die im Detail gebraucht wurden
    \item Evaluation
    \item Conclusion
    \item Introduction
    \item Abstract
  \end{enumerate}
  \begin{description}
    \item[Antwort: ] passt
  \end{description}
  \item Fokus auf DNA-Tile based Nanonetze - nur in-vivo, oder auch in-vitro (Sicherheitfaktor)
  \begin{description}
    \item[Antwort: ] siehe unten
  \end{description}
  \item Kommunikationsprotokolle Problem
  \begin{itemize}
    \item bislang nicht wirklich mehrere Protokolle für eine Technologie gefunden
    \item eher immer verschiedene Nenotechnologien, die dann auf eine bestimmte Art kommunizieren -> Kommunikationsprotkoll für den speziellen Fall
    \item Sicherheit Faktor jetzt schonmal diskutieren - vllt etwas früh (für Recherche)
    \item nicht mehr vorhandene Artikel in Wiss. Magazinen
  \end{itemize}
  \begin{description}
    \item[Antwort: ] andere Kommunikationsprotkolle auf unsere Technologie. Bad Actor nicht darauf eingehen, nochmal anders. In vitro, in vivo beides beachten. 6LowPAN, o.Ä. Protokolle COAB betrachten ... 1-3 aus der Forschung. Aus den Protokollen Anforderungen ableiten, die in NET-TAS simuliert werden können -> a-tam, k-tam ... wie gut werden die Anforderungen. Skript schreiben, für JSON Objekte, die in die Simulation eingefügt werden können.
  \end{description}
  \item Organisation: meine Wunschvorstellung: Festgelegter Tag/Uhrzeit alle 2, maximal 3 Wochen. Brauche Zeitdruck, Bringschuld. Im Meeting einmal kurz Zusammenfassen, was ich gemacht habe und was mein Plan für die nächsten Wochen ist. Wenn das nicht geht, dann würde ich das über email machen.
  \begin{description}
    \item[Antwort: ] Anmeldung zur Masterarbeit auf Homepage
  \end{description}
  \item ITM Logo (am besten pdf)
  \begin{description}
    \item[Antwort: ]
  \end{description}
  \item weitere Notizen:
  \begin{itemize}
    \item Unsetzung von Protokollen in DNA-Tile-basierten Nanosystemen
  \end{itemize}
\end{itemize}

\section{Besprechung 18-04-23}
  \begin{itemize}
    \item Frage: In-Vitro/In-Vivo in Grundlagen erklären? oder als Vorwissen erwarten?
    \begin{description}
      \item[Antwort:] Ja erklären, schräg setzen beim ersten Erwähnen allgemein
    \end{description}
    \item Frage: Simulation und JSON-Skript nochmal im Detail klären/unklar (Beispiel)
    \begin{description}
      \item[Antwort:] nettas.item.uni-luebeck.de -> ausprobieren für Skript
    \end{description}
    \item Frage: Grundlagen Struktur richtig? Was davon sollte eher in Related Works? Was fehlt?
    \begin{itemize}
      \item DNA
      \item Kommunikationsprotokolle
      \item Nanonetzwerke
      \begin{itemize}
        \item DNA-Self-Assembly
        \item DNA-based Nanonetze
        \item Nanonetzspezifische Kommunikationsprotkolle
      \end{itemize}
    \end{itemize}
    \begin{description}
      \item[Antwort:] self Assembly kann auch in DNA direkt
    \end{description}
    \item Frage: Allgemeine Struktur der Arbeit so richtig? Nach Leitfaden sollte "Simulationen" und "Evaluation" eher in den Hauptteil. Kann mir persönlich aber vorstellen, dass damit ein Kapitel 90\% der Arbeit ausmachen würde.
    \begin{description}
      \item[Antwort:] Related Work, Konzeption in den Hauptteil
    \end{description}
    \item Frage: Was genau soll auf die "Aufgabenstellung" Seite?
    \begin{description}
      \item[Antwort:]
    \end{description}
    \item Frage: Zitierung in Überschrift? Ja/Nein?
    \begin{description}
      \item[Antwort:] Nein
    \end{description}
  \end{itemize}

\section{Besprechung 04.05.23}
  \begin{itemize}
    \item Frage: Allgemein unsicher zur Struktur der Nanonetzwerke Section. Sollte ich immer wieder neue Sections mache, obwohl das ein großer Überschlag zum Thema Nanonetzwerke ist?
    \begin{description}
      \item[Antwort:] erst Grundlagen. Am Ende erst Kapitel Nanonetzwerke. In der Regel immer der gleiche Aufbau. Im Ganzen und im Kleinen
    \end{description}
    \item Frage: Grundlagen/histroische Einordnung verschmelzen bei mir aktuell. Wie viel Historie darf in Grundlagen sein?
    \begin{description}
      \item[Antwort:] Lass es in den Grundlagen.
    \end{description}
    \item Frage: Quelle des DNA Modell Bildes checken
    \begin{description}
      \item[Antwort:] den Author von wikimedia immer nachschlagen. Hier: Leyo
    \end{description}
    \item Frage: Was ist bessere Form? Lange Bildbeschreibung und dann nur verweis auf Bild im Fließtext? Oder kurze Bildbeschreibung und ausführlichere Beschreibung des Bildes im Fließtext? Oder beides?
    \begin{description}
      \item[Antwort:] Beides!
    \end{description}
    \item Frage: Latex Hilfe bei subfigures. Irgendwie wirkt das alles richtig für mich, aber LaTeX stirbt so einfach nur. 
    \begin{description}
      \item[Antwort:] GPT hilft hoffentlich
    \end{description}
    \item Frage: Wie zitiere ich (zitiere ich überhaupt?) "logische" Informationen? Bsp.: Merkmale von Nanonetzwerken (Viele Teilnehmer, Hohe Topologie, Heterogenität, Zeit-/Platzbeschränkungen, usw.) Ich kann diese Sachen ja nicht einfach behaupten. Wie zitiere ich die/finde Arbeiten, die diese Dinge in meinem Kontext definieren?
    \begin{description}
      \item[Antwort:] Allgemeine Arbeit suchen zum zitieren
    \end{description}
    \item Frage: Verwirrung von deutschen/englischen Anführungsstrichen. Allgemein unsicher wann Anführungsstriche notwendig und wann kursiv schreiben ausreicht.
    \begin{description}
      \item[Antwort:] Anführungsstriche bei Umgangssprache. Erklärung des zu definierenden Wort, schräg. Name Schräg. Abkürzung in Klammern dahinter nicht schräg. Definient, Definiendum, etc.
    \end{description}
    \item Frage: DX-/TX-Tiles Verbindung zur Holiday Junction unklar. Wofür brauche ich Holiday Junction, wenn ich auch X-Shaped DX-Tiles verwenden könnte?
    \begin{description}
      \item[Antwort:] Holliday Junction z.b. kleiner. Aber es ist eine möglichkeit, nicht die einzige
    \end{description}
    \item Frage: Muss ich bei Bilder, die ich leicht bearbeitet habe (Bsp: Farbfilter) irgendwas in der Zitierung ergänzen?
    \begin{description}
      \item[Antwort:] Autor fragen, ob ich das Bild verändern darf. Dann in der Bildbeschreibung angeben.
    \end{description}
    \item Frage: Tabelle aTAM usw. Error behandlung sinnvoll/richtig?
    \begin{description}
      \item[Antwort:]
    \end{description}
    \item Frage: Klarstellung mit Definitionen aus Skript/Buch-Draft wegen den richtigen Quellen.
    \begin{description}
      \item[Antwort:] nur Dinge Definieren, die am Ende auch in der Arbeit benötigt werden.
    \end{description}
    \item Frage: mir fehlt irgendwie eine klare Quelle für das kxk Proofreading
    \begin{description}
      \item[Antwort:] suchen!
    \end{description}
    \item Frage: Laufzeit und Abhängigkeit zur Größe des Tilesets von Self-Assemblies
    \begin{description}
      \item[Antwort:]Simulation dauert länger, realität nicht. sondern eher schneller, da es mehr stellen gibt, an denen sich Moleküle binden können.
    \end{description}
    \item Frage: Lebenszeit eines solchen Moleküls. Möglichkeit des "Aufräumens"?
    \begin{description}
      \item[Antwort:] in vitro ohne uv 15000 jahre. in-vivo durch milz ein paar Stunden
    \end{description}
    \item Frage: Quelle für das Nanonetzwerk Beispiel
    \begin{description}
      \item[Antwort:] Dis Lau
    \end{description}
    \item Frage: Wenn ich das OSI-Modell Bild in schön dem Bild nachempfinde, wie zitiere ich dann richtig?
    \begin{description}
      \item[Antwort:] primärquelle zitieren
    \end{description}
    \item Frage: Grundsatz Diskussion um Kommunikationsprotokolle in diesen Systemen. Wo setzt das Protokoll genau an? ist das Sensor -> Robotor Beispiel nicht eigentlich schon eine Art Protokoll?
    \begin{description}
      \item[Antwort:]Kommunikation im Draft definiert. Drum herum. klare Definition nochmal durchgehen.
    \end{description}
  \end{itemize}

\section{Besprechung 18.05.2023}
\begin{itemize}
  \item Allgemeines zur Kommunikation
  \begin{description}
    \item[Antwort:] etwas mehr an der VL halten erstmal. Nicht stumpf einzelne Protokolle übernehmen. Stapel anschauen. Was gibt es schon, was nicht.
  \end{description}
\end{itemize}

\section{Besprechung Skelett}
  \begin{itemize}
    \item Frage:
    \begin{description}
      \item[Antwort:]
    \end{description}
  \end{itemize}
