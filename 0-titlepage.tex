%!TEX root = thesis.tex
%---------------------------------------------------------------------------
% Frontpage
%---------------------------------------------------------------------------

% Die Richtline zum Aufbau des Deckblatts von Bachelor- und Masterarbeiten
% findet sich hier:
% @see: http://www.uni-luebeck.de/fileadmin/uzl_ssc/PDF-Dateien/Richtlinie-Deckblatt-MINT-Abschlussarbeit-2012-10-18.pdf

\newcommand{\titlepageskip}{\vskip 20pt}

% @see: http://tex.stackexchange.com/questions/31705/different-margins-for-title-page
\newgeometry{top=1in,bottom=1in,right=1in,left=1.2in}
\begin{titlepage}

\title{Übertragung herkömmlicher Netzwerkmechanismen auf DNA-basierte Nanonetzwerke}
\author{Andreas Waldner}

{\Large
	% 1. Offizielles Logo des Instituts, an dem die Arbeit angesiedelt ist. (Das offizielle Logo
	% enthält das Siegel der Universität zusammen mit dem Text "Universität zu Lübeck"
	% und darunter den Namen des Instituts.) Dieses Logo ist bei den Instituten zu
	% bekommen. Das Logo muss oben links platziert werden.
	\includegraphics[width=80mm]{Logo_Inst_Telematik_cropped}
	\vskip 44pt

	% 2. Optional: Noch einmal Name des Instituts und Angabe der Direktorin oder des
	% Direktors des Instituts.

	% 3. Titel der Arbeit in deutscher Sprache und ebenfalls in englischer Sprache. Dabei soll
	% die Sprache, in der die Arbeit verfasst wurde, als erste angeführt werden; die andere
	% Sprache kann weniger prominent dargestellt werden.
	% Auch bei englischsprachigen Studiengängen sollen die Titelblätter auf Deutsch sein.
	\textbf{\LARGE Übertragung herkömmlicher Netzwerkmechanismen auf DNA-Tile-basierte Nanonetzwerke}\\
	\textbf{\LARGE Transfer of conventional network mechanisms to DNA-Tile-based nanonetworks}\\
	\titlepageskip
	% 4. Der Text "Bachelorarbeit" oder "Masterarbeit" (nicht "Bachelor-Arbeit" oder "Master-Arbeit").
	%\textbf{Bachelorarbeit}
	\textbf{Masterarbeit}

	\titlepageskip
	%5. Der Text "im Rahmen des Studiengangs"
	im Rahmen des Studiengangs\\
	%6. Der ausgeschriebene Name des Studiengangs (also beispielsweise "Informatik"
	%oder "Molecular Life Science", hingegen nicht "Bioinformatik" oder "MLS")
	\textbf{Informatik}\\
	%7. Der Text "der Universität zu Lübeck"
	der Universit"at zu L"ubeck

	\titlepageskip
	%8. Der Text "Vorgelegt von" und der Name der Studentin oder des Studenten
	vorgelegt von\\
	\textbf{Andreas Waldner}

	\titlepageskip
	%9. Der Text "Ausgegeben und betreut von"
	ausgegeben und betreut von\\
	%10. Der Name der ersten Prüferin oder des ersten Prüfers. Dies ist immer gleichzeitig
	%die Betreuerin oder der Betreuer im Sinne der Prüfungsordnung.
	\textbf{Dr. rer. nat. Florian-Lennert Lau}

	% Diesen Teil entfernen, wenn die Arbeit KEINEN Unterstützer hatte
	\titlepageskip
	

	\vfill
	%13. Der Text "Lübeck, den" und das Abgabedatum.
	{
		Lübeck, den 17. Oktober 2023
	}

	% Diesen Teil entfernen, wenn "Im Focus das Leben" nicht drauf stehen soll
	%14. Optional der Text "Im Focus das Leben".
	{
		\titlepageskip
		Im Focus das Leben
	}
}
\end{titlepage}
\restoregeometry

\cleardoublepage

% Erklaerung
\newpage
\chapter*{Erkl"arung}

Ich versichere an Eides statt, die vorliegende Arbeit selbstständig und nur unter Benutzung
der angegebenen Hilfsmittel angefertigt zu haben.

\vspace*{3cm}
Lübeck, den 17. Oktober 2023

\thispagestyle{empty}
\cleardoublepage


% Kurzfassung und Abstract

\chapter*{Kurzfassung}

In der heutigen technologischen Landschaft zeichnet sich ein Trend zur kontinuierlichen Miniaturisierung ab, wobei wir schon seit einigen Jahren in die Nanoebene eintauchen. Obwohl Nanomaterialien und -technologien bereits in zahlreichen Bereichen Anwendung finden, birgt diese Dimension immer noch ein großes und unausgeschöpftes Potenzial. Diese Arbeit beschäftigt sich mit der Anwendung herkömmlicher Mechanismen aus Netzwerken und Kommunikationsprotokollen auf DNA-basierten Nanonetzwerken. Während konventionelle Netzwerke eine breite Palette von Mechanismen und Strategien bieten, besteht das Ziel dieser Arbeit darin, deren Anwendbarkeit und Relevanz im Gebiet der Nanonetzwerke zu erforschen. Um Simulation und Analyse dieser Mechanismen zu optimieren, wurde ein Python-Skript entwickelt, das Tilesets generieren oder modifizieren kann. Die Ergebnisse dieser Arbeit bieten wertvolle Einblicke in das Potenzial von DNA-basierten Nanonetzwerken. Auch werden Grenzen und Probleme dieser Technologie analysiert und aufgezeigt.

%
\vskip 3cm
%

\section*{\huge Abstract}

In today's technological landscape, a trend towards miniaturization is emerging, and we have been immersed in the nanoscale for several years now. Although nanomaterials and nanotechnologies already have applications in numerous fields, this dimension still holds great and untapped potential. This work focuses on the application of conventional mechanisms from networks and communication protocols to DNA-based nanonetworks. While conventional networks offer a wide range of mechanisms and strategies, the goal of this work is to explore their applicability and relevance in the field of nanonetworks. To optimize simulation and analysis of these mechanisms, a Python script was developed that can generate or modify tile sets. The results of this work provide valuable insights into the potential of DNA-based nanonetworks. Also, limitations and problems of this technology are analyzed and highlighted.

\thispagestyle{empty}
\cleardoublepage